% введение

Термин Умный город появился относительно недавно, и определенного конкретного определения этому понятию нет. Но, все-таки  эксперты сошлись в том, что главный источник управления <<смарт сити>> – данные о населении. 

Умный город (smart city) - это стратегическая концепция по развитию городского пространства, подразумевающая совместное использование информационно - коммуникационных технологий (ИКТ) и решений Интернета вещей (IoT) для управления городской инфраструктурой. К нему относятся транспортные системы, водопроводные каналы, медицинские организации, системы переработки отходов и множество других общественных служб. \cite{Harrison}

Главная идея системы Умный город - организация информационного пространства, которое содержит в себе данные о работе контролируемых объектов (счетчиков тепловой и электрической энергии, лифтов, электротехнического оборудования, различных технических средств безопасности и т.д.). На любом расстоянии можно управлять объектами в режиме реального времени, вне зависимости от места расположения объектов и центрального управляющего пункта в городе.

Удорожание тарифов на тепловую энергию, горячую и холодную воду приводит к тому, что потребители всё больше задумываются о точной и своевременной оценке количества потреблённых ресурсов. Повсеместная установка приборов учёта является сегодня одним из приоритетных направлений реформирования ЖКХ. Однако, кроме монтажа счётчика, необходимо обеспечить возможность оперативного и регулярного снятия показаний с него. Пока счётчиков мало, эту операцию можно проводить и вручную, но как только количество узлов учёта начинает исчисляться десятками и сотнями, возникает задача создания системы автоматического сбора показаний. Такая диспетчеризация позволяет не только оперативно собирать данные, но и проводить всесторонний анализ работы теплосетей (например, выявлять неисправности).

Есть большая потребность в обработке поступающих данных в реальном времени с целью выявления аварийных и предаварийных ситуаций, нарушений работоспособности счетчиков водоснабжения, электроэнергии, нарушений режимов теплоснабжения. Такие потребности возникают и у потребителей, и у эксплуатирующих организаций, и у поставщиков тепловой энергии.

Проблема: Некорректные начисления по оплате \cite{Almanah}
\begin{itemize}
	\item Неверная работа квартирных приборов учета (неисправности, воровство); 
	\item поставка ресурсов ненадлежащего качества (недогрев, перегрев, некачественная электроэнергия);
	\item Некорректная работа регулирующего теплового оборудования; 
	\item Отсутствие технической возможности проводного соединения с квартирными приборами учета.
\end{itemize}

Решение: 
\begin{enumerate}
	\item Создание единой облачной базы данных; 
	\item Автоматизированный анализ
\end{enumerate}

\begin{itemize}
	\item Квартирные ПУ с импульсными выходами, протоколом передачи данных M-Bus, беспроводные ПУ с протоколом LoRa-WAN;
	\item Домовые электросчетчики с архивированием данных по качеству электроэнергии;
	\item Формирование базы показаний с единым форматом данных; конвертирование архивов ПУ различных производителей;
	\item Автоматизированный анализ данных.
\end{itemize}

Задачи, решаемые в ходе работы (в соответствии с заданием на НИР):
 \begin{enumerate}
 	\item Разработка автоматизированной системы учета потребления ресурсов;
	\item Анализ показаний квартирных и домовых приборов учета; 
	\item Алгоритмы поиска проблемных счетчиков электроэнергии и водоснабжения для разных типов ошибок;
	\item Описание программного обеспечения, позволяющего контролировать приборы учета.
\end{enumerate}
