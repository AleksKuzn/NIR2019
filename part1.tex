% первая часть

\section{Система учета ресурсов}
Система учета ресурсов предназначена для считывания, мониторинга и работы с   данными домовых счетчиков учета ресурсов. Данные со счетчиков попадают на сервер баз данных, в программе диспетчера и администратора данные отображаются, считаются, формируются отчеты. \cite{NK}

Система учета ресурсов состоит из

\begin{itemize}
	\item Сервера баз данных;
	\item Контроллеры приборов учёта;
	\item Преобразователей интерфейса;
	\item Домовой сервер (Raspberry Pi);
	\item Службы сбора данных;
	\item Управляющей программы(администратора и диспетчера).
\end{itemize}
Администратор системы должен контролировать состояния на панеле мониторинга, добавлять новые счетчики, составлять отчеты, проводить аналитику. 

Организация системы сбора показаний.

\begin{enumerate}
	\item Опросом показаний занимается программа, запущенная на домовом сервере (Raspberry Pi).  Она должна извлекать данные о требуемых к опросу счетчиках из текстового файла (приложение 1.1), производить опрос и сохранять показания в отдельный текстовый файл (приложение 1.2). Опрос показаний должен производится 1 раз  в час (например в 05 минут каждого часа).
	\item Служба сбора данных производит непосредственный опрос всех Raspberry Pi. Эта программа устанавливается на сервер баз данных и производит выгрузку файлов data.txt  с домовых серверов в единую базу данных (FireBird, PostgreSQL) (приложение 2). После занесения данных в основную базу необходимо очистить файл data.txt.  Если в основной базе изменилась информация о счетчиках (порт, адрес КПУ или номер клеммы) данная программа должна загрузить новый файл counter.txt.
	\item Раз в сутки необходимо синхронизировать время домового сервера и сервера БД (это может быть отдельная программа или одна из функций программы из п.2).	
\end{enumerate}

\textbf{квартирный учет ресурсов}
 
 В квартирном учете применена проводная система снятия показаний. Для снятия показаний приборов учета с импульсным выходом применяется универсальный счетчик СКАУТ-КПУ, поддерживающий 12 импульсных входов. Приборы с протоколами передачи данных RS485, ModBUS и другими подключаются к плате сопряжения протоколов СКАУТ

\textbf{домовой учет ресурсов}

Домовые приборы учета с импульсным выходом применяются в комплекте со счетчиком импульсов СКАУТ-КПУ. При наличии у приборов внутреннего архива, опрос приборов и промежуточное хранение файлов архива осуществляется микрокомпьютером СКАУТ-Базовый

\subsection{монтаж: установка оборудования}

\begin{enumerate}
	\item Установить комплекс СКАУТ базовый в защитный шкаф
	\item Установить контроллер приборов учета
	\item Установить "умные" счетчики, с системой телеметрии на каждый вид ресурсов (газ, электроэнергия, вода, тепло)
	\item Проложить коммутационные кабели и подключить оборудование к сети связи
	\item Установить программное обеспечение на компьютер сотрудника УК
	\item Создать базы данных по счетчикам и адресам; провести пуско-наладочные работы
	\item Провести инструктаж сотрудников УК по использованию системы учета ресурсов
\end{enumerate}