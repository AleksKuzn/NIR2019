% третья часть

\section{Алгоритмы поиска проблемных счетчиков}
Коммунальные компании постоянно сталкиваются с неучтенными расходами воды.
Чтобы справиться с утечками и сократить неучтенные расходы воды, необходимо четко представлять себе ситуацию в распределительной сети. \cite{smartcity}

Наличие нужных данных в нужное время значительно облегчает борьбу с потерей и неучтенными расходами воды и повышает эффективность этих действий. Автоматизированная система учета отображает реальное состояние распределительной сети, помогает выявить различные типы неучтенных расходов и сократить потери воды. 

Рассмотрим проблему, когда от счетчика не поступает импульс. Есть два варианта:
\begin{itemize}
	\item счетчик не исправен, по какой-то причине не крутится роликовый индикатор;
	\item счетчик работает, но импульсы от него не поступают.
\end{itemize}

Ошибка №1. Проблемы со счетчиком ХВС. 

Потребление ХВС = 0 за сутки, при этом потребление ГВС > 0 за сутки.

Алгоритм поиска:

Находим разность между показаниями счетчиков ХВС и ГВС, за текущее число и за вчерашнее.
Оставляем значения счетчиков ХВС и ГВС, где разность ХВС = 0.
Убираем значения счетчиков ХВС и ГВС, где разность ГВС = 0.

Оставшиеся счетчики ХВС являются проблемными (можно сравнивать не за сутки, а за 3 или 5).

Ошибка №2. Проблемы со счетчиком ГВС.

Потребление ГВС = 0 за неделю, при этом потребление ХВС > 0 за неделю.

Алгоритм поиска:

Находим разность между показаниями счетчиков ХВС и ГВС неделю назад и за текущее число.
Оставляем значения счетчиков ХВС и ГВС, где разность ГВС = 0.
Убираем значения счетчиков ХВС и ГВС, где разность ХВС = 0.

Получили счетчики, которые попадают в поле подозрения.
Нужно провести отбор, возможно горячей водой просто не пользуются.

Убираем счетчики, у которых значения ГВС < 5 и значение ХВС < 10.
Оставшиеся счетчики ГВС являются проблемными и требуют ручной проверки.

Ошибка №3. Проблемы со счетчиком ХВС и ГВС.

Потребление ХВС = 0, ГВС = 0 за неделю, при этом потребление Э  >= Ср.Ар. за предыдущую неделю.

Алгоритм поиска:

Находим разность между показаниями счетчиков ХВС и ГВС неделю назад и за текущее число.
Оставляем значения счетчиков ХВС и ГВС, где разность ГВС = 0 и ХВС = 0.
Находим среднее арифметическое потребление электричества за предыдущую неделю.
Если потребление электроэнергии за неделю больше чем среднее арифметическое значение, то счетчики попадают в список проблемных счетчиков.

Ошибка №4. Проблемы с электросчетчиком.

Потребление Электричества = 0 за сутки, при этом потребление ХВС > 0 и/или ГВС > 0 за сутки и более.