% четвертая часть

\section{Описание программного обеспечения}

Система учета ресурсов предназначена для считывания, мониторинга и работы с данными домовых счетчиков учета ресурсов. Данные со счетчиков попадают на сервер баз данных, в программе диспетчера и администратора данные отображаются, считаются, формируются отчеты.

Полный функционал программного обеспечения:
\begin{itemize}
\item управление адресами и объектами установки ПУ;
\item управление приборами учета;
\item управление абонентами; 
\item просмотр показаний ПУ за выбранный интервал времени; 
\item расчет потребления энергоресурсов по основным показаниям ПУ за указанный интервал времени; 
\item просмотр детальной информации по потреблению энергоресурсов конкретного ПУ с выводом графика потребления; 
\item предоставление сведений об аварийных и нештатных ситуациях ПУ; 
\item экспорт полученных данных в другие форматы, вывод на печать;
\item поиск.
\end{itemize}

Программа включает 9 основных разделов.
\begin{enumerate}
	\item Раздел - Показания приборов
	\item Раздел - Состояние приборов учета
	\item Раздел - Адреса
	\item Раздел - Абоненты
	\item Раздел - Приборы учета
	\item Раздел - Лицевые счета
	\item Раздел - Отчеты
	\item Раздел - Поиск
	\item Раздел - Оповещение об ошибках и нештатных ситуациях
\end{enumerate}

Система должна быть защищенной и поэтому используются локальные данные. Сервер, база данных на постгрессе и программы на компьютерах администраторов для управления. 

\textbf{Цели:} Сбор данных и мониторинг потребления ресурсов.
 
\textbf{Задачи:}  Создать защищенную и удобную систему  для сбора данных потребляемых населением.

Программа для администратора: мониторинг, добавление/ изменение/ удаление счетчиков, составление отчетов. 

Функции программы диспетчера:

Ведение списка пользователей и управление их полномочиями;
Конфигурация для сохранения, изменения и отображать данных о подключении счетчика к конкретной КПУ (порт, адрес, клемма, коэффициент- цены импульса). 
\begin{itemize}
	\item Ведение служебных справочников (список адресов, типов приборов и т.д.);
	\item Ввод и редактирование данных о подключенных контроллерах;
	\begin{enumerate}
		\item Создание прикрепленных индивидуальных приборов учёта с указанием:
		\begin{enumerate}
			\item номера клеммы; 
			\item цены импульса; 
			\item типа прибора;
			\item серийного номера; 
			\item единиц измерения;
			\item начального показания;
			\item номера квартиры;
			\item ФИО собственника;
		\end{enumerate}
		\item Для счётчиков требующих проверки указать дату предыдущей и следующей поверки;
		\item Групповые операции с устройствами;
	\end{enumerate}
	\item Просмотр показаний приборов учёта в табличном и графическом виде;
	\begin{enumerate}
		\item Отображение сведений о наличии или отсутствии показаний за выбранный период для выбранных устройств;
	\end{enumerate}
	\item Формирование отчетов по потреблению ресурсов – ежемесячный отчет установленного формата для управляющей компании/ресурсоснабжающей организации;
	\item Сравнение суммарного квартирного потребления и общедомового (при наличии возможности).
	
\end{itemize}